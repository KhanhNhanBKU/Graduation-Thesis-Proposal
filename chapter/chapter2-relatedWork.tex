\chapter{Related Work}
    CheckList
      [ ] Add image for glove base approach about this
      [ ] Add image about process of this

    Nowadays, research works related to the problem of converting sign language into text 
    have been proposed by many researchers from all over the world, from many different 
    approaches and perspectives.

    “Hand talk-a sign language recognition based on
    accelerometer and semi data” this paper introduces American
    Sign Language conventions. It is part of the “deaf culture” and
    includes its own system of puns, inside jokes, etc. It is very
    difficult to understand understanding someone speaking
    Japanese by English speaker. The sign language of Sweden is
    very difficult to understand by the speaker of ASL. ASL
    consists of approximately 6000 gestures of common words
    with spelling using finger used to communicate obscure words
    or proper nouns.

    “Hand gesture recognition and voice conversion system for
    dumb people” proposed lower the communication gap
    between the mute community and additionally the standard
    world. The projected methodology interprets language into
    speech. The system overcomes the necessary time difficulties
    of dumb people and improves their manner. Compared with
    existing system the projected arrangement is simple as well as
    compact and is possible to carry to any places. This system
    converts the language in associate text into voice that's well
    explicable by blind and ancient people. The language
    interprets into some text kind displayed on the digital display
    screen, to facilitate the deaf people likewise. In world
    applications, this system is helpful for deaf and dumb of us
    those cannot communicate with ancient person.

    In which, two main approaches can be mentioned as follows:
    \begin{itemize}
      \item Glove based approaches:
      With this approach, it requires deaf and mute people to wearing a sensor glove. When user
      has any different action or gesture, these sensor will be recorded. After that, data from
      sensor will analyze by analyzer component and return the output for user.
      \item Vision based approaches:
      With this approach, image processing algorithms will be applied to be able to determine
      hand position, gestures and movements of the hand. The user will not have to wear necessary
      equipment like glove based approaches, which is convenient for user. However, with using
      library or algorithms of image processing, we need to deal with worst quality output, which is
      greatly affected by this algorithms.
    \end{itemize}

    With both approaches above, there is has some problems, that is, they can only recognize 
    a very small number of words. These words are mostly words with different hand shapes 
    that can be classified like that. However, in sign language, there will be many words 
    that use the same hand shape but will differ in many characteristics, such as position 
    and orientation. To our knowledge, there is currently no model that can handle the 
    conversion of sign language flexibly and conveniently for the deaf-mute, helping them 
    to communicate effectively. natural to the common man. Therefore, by applying appropriate 
    technologies, the authors carry out this graduation thesis with the goal of breaking down 
    the barriers between deaf-mute people and normal people, helping them to become self-sufficient. 
    more confident in daily communication.
