\chapter{Related Work}
  CheckList: 
    [X] Sơ lược ý chính
    [X] Điều chỉnh
    [.] Translate
    [ ] Complete

    Nowadays, research works related to the problem of converting sign language into text 
    have been proposed by many researchers from all over the world, from many different 
    approaches and perspectives. In which, two main approaches can be mentioned as follows:
      - Glove based approaches:
        With this approach, it requires deaf and mute people to wearing a sensor glove. When user
        has any different action or gesture, these sensor will be recorded. After that, data from
        sensor will analyze by analyzer component and return the output for user.
      - Vision based approaches:
        With this approach, image processing algorithms will be applied to be able to determine
        hand position, gestures and movements of the hand. The user will not have to wear necessary
        equipment like glove based approaches, which is convenient for user. However, with using
        library or algorithms of image processing, we need to deal with worst quality output, which is
        greatly affected by this algorithms.
    With both approaches above, there is has some problems, that is, they can only recognize 
    a very small number of words. These words are mostly words with different hand shapes 
    that can be classified like that. However, in sign language, there will be many words 
    that use the same hand shape but will differ in many characteristics, such as position 
    and orientation. To our knowledge, there is currently no model that can handle the 
    conversion of sign language flexibly and conveniently for the deaf-mute, helping them 
    to communicate effectively. natural to the common man. Therefore, by applying appropriate 
    technologies, the authors carry out this graduation thesis with the goal of breaking down 
    the barriers between deaf-mute people and normal people, helping them to become self-sufficient. 
    more confident in daily communication.



  Ngày nay, các công trình nghiên cứu liên quan đến vấn đề chuyển đổi ngôn ngữ
  ký hiệu thành văn bản đã được nhiều nhà nghiên cứu từ khắp nơi trên thế giới đề xuất
  , theo nhiều hướng tiếp cận và góc nhìn khác nhau. Trong đó có thể kể đến 2 hướng
  tiếp cận chính như sau:
    - Hướng tiếp cận sử dụng găng tay cảm biến:
        Đây là hướng tiếp cận mà người sử dụng sẽ đeo 1 chiếc găng tay được trang bị
        các cảm biến chuyển động chuyên dùng. Khi người sử dụng có các hành động hay
        cử chỉ khác nhau sẽ được các cảm biến này ghi nhận, sau đó qua một bộ phân
        tích và sẽ trả về kết quả cho người dùng
    -Hướng tiếp cận sử dụng xử lý hình ảnh:
        Trong hướng tiếp cận này, các thuật toán về xử lý hình ảnh sẽ được áp dụng để
        có thể xác định được vị trí bàn tay, các cử chỉ, chuyển động của bản tay như thế nào
        . Người sử dụng sẽ không phải mang các trang bị cần thiết như hướng tiếp cận sử 
        dụng găng tay, thuận tiện cho người sử dụng. Tuy nhiên,
        độ hiệu quả của các thuật toán xử lý ảnh hướng rất nhiều đến chất lượng đầu ra.

  Với cả hai cách tiếp cận trên đều có một đặc điểm chung, đó là đều chỉ có thể
  nhận diện được một số lượng rất ít từ vựng. Các từ vựng này hầu hết là các từ có sự khác nhau
  về hình dạng bàn tay thì mới có thể phân loại được như thế. Tuy nhiên, trong ngôn ngữ ký hiệu
  , sẽ có rất nhiều từ sử dụng chung một hình dạng bàn tay nhưng sẽ khác nhau về nhiều đặc điểm
  , ví dụ như vị trí và hướng. Theo hiểu biết của chúng em thì hiện nay vẫn chưa có một mô hình nào
  có thể xử lý được việc chuyển đổi ngôn ngữ ký hiệu một cách linh hoạt và thuận tiện cho người câm-điếc,
  giúp họ có thể giao tiếp được một cách tự nhiên với người bình thường. Chính vì thế, bằng cách vận dụng những công nghệ
  phù hợp, nhóm tác giả tiến hành thực hiện đề tài luận văn tốt nghiệp này hướng đến mục tiêu phá bỏ các rào cản
  giữa người câm-điếc và người bình thường, giúp họ tự tin hơn trong việc giao tiếp hằng ngày.



In this category requires signers to wear a sensor glove or a
colored glove. The task will be simplified during
segmentation process by wearing glove. The drawback of this
approach is that the signer has to wear the sensor hardware
along with the glove during the operation of the system.

    + Hướng tiếp cận sử dụng xử lý hình ảnh
