% This is a copy version from https://github.com/thanhhungqb/thesis-template
% Please do not modified this project, when you want to start writing, make a clone of it for your own (Please read README.md)

\documentclass[12pt,a4paper,oneside]{book} % twoside for draf

%\usepackage{babel}
\usepackage[utf8]{vietnam}
\usepackage[utf8]{inputenc}
\usepackage[vietnamese,english]{babel}

\usepackage{tipa}
\usepackage{amssymb}
\usepackage{graphicx, wrapfig}
\usepackage{subcaption}
\usepackage{booktabs}
\usepackage{mathptmx}	% same Time New Roma
\usepackage{amsmath}
\usepackage{amssymb}
\usepackage{amsfonts}
\usepackage{sty/ipa}
%\renewcommand{\rmdefault}{phv} % Arial
%\renewcommand{\sfdefault}{phv} % Arial
\usepackage{array}
\newcolumntype{P}[1]{>{\centering\arraybackslash}p{#1}}
\newcolumntype{M}[1]{>{\centering\arraybackslash}m{#1}}
\usepackage{fancyhdr}
\usepackage{multirow}
\usepackage{algorithm2e}
\usepackage{hyperref}
\usepackage{float}

\usepackage{sty/bkthesis}

% To numbering figure continuously
\usepackage{chngcntr}
\counterwithout{figure}{chapter}

\setlength{\parskip}{1em}
\renewcommand{\baselinestretch}{1.2}


\graphicspath{ {image/} }
\setcounter{secnumdepth}{2}
\crname{GRADUATION THESIS PROPOSAL}
% \ctname{NHẬN DIỆN VẬT THỂ TRONG ẢNH\\NHẬN DIỆN HƯỚNG NHÌN TRONG ẢNH}
\ctname{USING MACHINE LEARNING METHODS\\IN TRANSLATING SIGN LANGUAGE\\INTO VIETNAMESE}
\cstuname{
Student: Võ Tuấn Khanh (1810220)\\Nguyễn Trí Nhân (1810390)}
\csSupervise{Assoc. Prof. Quan Thanh Tho}

\cttime{December 2021}

\thesislayout

\begin{document}
%-	Bìa cứng - màu xanh dương, chữ mạ vàng (xem mẫu đính kèm)
%-	Trang tên (tờ lót): chất liệu giấy, nội dung giống như bìa LV
%-	Ở gáy LV: in nhan đề LV (có thể in tóm tắt nếu nhan đề quá dài), size 15 – 17
%-	Phiếu Nhiệm vụ LV, chấm điểm Hướng dẫn & Phản biện (đã ký): nhận từ GVHD & GVPB sau khi bảo vệ (theo lịch hẹn).
%-	Lời cam đoan
%-	Lời cảm ơn/ Lời ngỏ
%-	Tóm tắt LV
%-	Mục lục
%-	Danh mục, bảng biểu, hình ảnh, ... (nếu có)
%-	Nội dung LV
%-	Danh mục TL tham khảo
%-	Phụ lục (nếu có)

\coverpage

\frontmatter


\begin{declaration}
	TODO: Viết sơ sơ về việc nội dung báo cáo không phải là false, ăn cắp này kia nọ

	ví dụ:

	Nhận diện hướng nhìn trong ảnh (Nhận diện vật thể trong ảnh) không phải là một đề tài mới nhưng vẫn là một thách thức bởi: trong các ứng dụng: việc nhận diện hướng nhìn của con người qua hình ảnh đòi hỏi kết quả chính xác cao, ở Việt Nam, hiện tại không  thực sự có nhiều nghiên cứu chuyên sâu về đề tài. Trong quá trình nghiên cứu đề tài có rất nhiều kiến thức không nằm trong chương trình giảng dạy ở bậc Đại học tuy vậy chúng tôi xin cam đoan đây là công trình nghiên cứu của riêng tôi dưới sự hướng dẫn của tiến sĩ Nguyễn Đức Dũng. Nội dung nghiên cứu và các kết quả đều là trung thực và chưa từng được công bố trước đây. Các số liệu được sử dụng cho quá trình phân tích, nhận xét được chính tôi thu thập từ nhiều nguồn khác nhau và sẽ được ghi rõ trong phần tài liệu tham khảo.
 	
 	Ngoài ra, tôi cũng có sử dụng một số nhận xét, đánh giá và số liệu của các tác giả khác, cơ quan tổ chức khác. Tất cả đều có trích dẫn và chú thích nguồn gốc.
 	
 	Nếu phát hiện có bất kì sự gian lận nào, tôi xin hoàn toàn chịu trách nhiệm về nội dung luận văn của mình. Trường đại học Bách Khoa thành phố Hồ Chí Minh không liên quan đến những vi phạm tác quyền, bản quyền do tôi gây ra trong quá trình thực hiện.

\end{declaration}

\begin{acknowledgments}
	TODO: Viết sau cùng -> về việc cảm ơn này kia

	ví dụ:

	Để hoàn thành kì đề cương luận văn này, tôi tỏ lòng biết ơn sâu sắc đến tiến sĩ Nguyễn Đức Dũng đã hướng dẫn tận tình trong suốt quá trình nghiên cứu.
	
	Chúng tôi chân thành cám ơn quý thầy, cô trong khoa Khoa Học Và Kỹ Thuật Máy Tính, trường đại học Bách Khoa thành phố Hồ Chí Minh đã tận tình truyền đạt kiến thức trong những năm chúng tôi học tập ở trường. Với vốn kiến thức tích lũy được trong suốt quá trình học tập không chỉ là nền tảng cho quá trình nghiên cứu mà còn là hành trang để bước vào đời một cách tự tin.

	Cuối cùng, tôi xin chúc quý thầy, cô dồi dào sức khỏe và thành công trong sự nghiệp cao quý.
	
\end{acknowledgments}
	
\begin{abstract}
	TODO: Viết sau cùng
	
	ví dụ:

	Nội dung chính của luận văn nhằm tìm hiểu, nghiên cứu xây dựng hệ thống nhận diện hướng nhìn thông qua ảnh chụp dựa trên những công trình, công nghệ mới được nghiên cứu và phát triển trong những năm gần đây của lĩnh vực Deep Learning. Trong quá trình nghiên cứu, tôi đã  tiến hành tổng hợp, đánh giá ưu và nhược điểm của cách phương pháp, công nghệ đã và đang được nghiên cứu, sử dụng. Tiếp cận vấn đề theo nhiều hướng khác nhau, tôi thực hiện một số phương pháp sử dụng học sâu (CNN) để phát hiện hướng nhìn của con người qua hình ảnh. Bên cạnh việc hoàn thành nội dung của đề tài, nhóm chúng tôi đã nghiên cứu thêm một số phần để từ đó đặt nền móng cho các nghiên cứu sau này. Phần còn lại của luận văn tập trung vào việc đánh giá mô hình, kết quả đạt được, đồng thời phân tích ưu nhược điểm của mô hình thực hiện và thảo luận những vấn đề mà mô hình còn gặp phải. Cuối cùng, nhóm chúng tôi đề xuất hướng phát triển tiếp theo của đề tài trong tương lai.
\end{abstract}	


\tableofcontents
% \listofsymbols
\listoffigures
\listoftables
%\listofalgorithms


\mainmatter

\fancyhead{}  % Clears all page headers and footers
%\rhead{\thepage}  % Sets the right side header to show the page number
%\lhead{}  % Clears the left side page header
%\fancyfoot[positions]{footer}
\renewcommand{\footrulewidth}{0.4pt}

\pagestyle{fancy}  % Finally, use the "fancy" page style to implement the FancyHdr headers
%

\chapter{Introduction}
	
\section{Problem statement}

"Each deaf person is a separate world, and they feel more self-deprecating and alone when they do not interact and share with others. They still have the desire to contribute to society", said Mr. Do Hoang Thai Anh, Vice Chairman of the Hanoi Deaf Association \cite{MoCanhCuaHyVong}.

Language is a universal key that not only connects people but also builds up our society. Any disability that affects the ability to communicate is a significant disadvantage, especially for people with disabilities. They cannot integrate, have fun, learn, and communicate like ordinary people because they cannot express their thoughts, ideas, and desires to develop society as we do. That burden usually makes them fall into poverty, live a dependent life, and be exploited, apart from society. Hence, it is challenging for them to have beautiful lives.

In 2020, Vietnam had more than 2.5 million people who are deaf and mute, yet, only a tiny portion of them took part in education, had the chance to be understood, and integrated with society \cite{ThieuPhienDich}.

According to UNICEF, Households with members with disabilities are often poorer, children with disabilities are at risk of having less education than their peers, and employment opportunities for people with disabilities are also lower than those without disabilities. Even though people with disabilities are beneficiaries of the policy, and poverty is not a burden to accessing health facilities, very few people with disabilities (2.3\%) have access to functional rehabilitation services when being sick or injured. Besides, there still exist inequalities in living standards and social participation for people with disabilities. Many organizations are founded to support, help, and create better living conditions for people with disabilities to develop. However, this work still has many difficulties and inadequacies as there is no formal school or class. Moreover, there is no specific profession for this group of people, and the number of translators who know sign language is insufficient, while they take an essential role in helping the people with disabilities connect with society.

A quote from Cavett Robert, "Life is a grindstone, and whether it grinds you down or polishes you up is for you and you alone to decide." However, it is challenging for these people to go to school and have an excellent education. They have their desires and dreams, but our resources and efforts are not enough to make them a polished grindstone. Furthermore, sign language shares the same property as any other spoken language; each different region and territory has a different way of expressing sign language. These unseen differences make communication, self-expression, and information exchange even more complex and challenging for humanity.

In short, we must admit that understanding and breaking the language barrier is extremely necessary and urgent because the deaf and mute, like many other ordinary people, deserve to be assisted, understood, and acknowledged. Furthermore, we believe our system is the resolve to problems of the deaf and hard of hearing.

\section{Goals of the thesis}

% Đề tài này hướng đến mục tiêu là nghiên cứu, hiểu và hiện thực một số giải pháp để có thể chuyển đổi từ ngôn ngữ ký hiệu sang tiếng Việt. Trong đó, hệ thống phải có khả năng thực hiện việc nhận video từ camera được gắn trên nón, thực hiện các giải thuật và xuất ra chữ trên màn hình điện thoại.
% 
% Từ mục tiêu tổng quát trên, tác giả sẽ lần lượt giải quyết các vấn đề sau để đưa ra một giải pháp thiết kế và hiện thực một kiến trúc hệ thống giải quyết được bài toán của đề tài:

the thesis aims to research, understand, and implement solutions to convert from sign language to Vietnamese. In particular, the system must receive a queue of images from the camera mounted on the hat, use the implemented algorithms to process and display text on the phone screen.

We can solve the above problem by breaking it down into smaller ones listed below. With each issue, we will give our solution and architect a system that can solve the whole problem of the thesis.
 
\begin{itemize}
	\item Search and collect data on sign language, conduct evaluation, classification, and normalization of data.
	\item Find out the approaches that have been implemented.
	\item Design architecture of the model
	\item Plan to implement, develop a sign language conversion system
	\item Build an application that users can utilize
\end{itemize}


\section{Scopes of the thesis}

% TLDR: In this case study, we will build a system including an app and camera module to translate at least 100 words from sign language into Vietnamese.

In this case study, we will build a system including an app and camera module to translate sign language into Vietnamese. Because of the limited time, the scope of the study is also limited as follows:

\begin{itemize}
	\item The system can only translate Vietnamese words
	\item The system can only recognize the words that it has been trained with
\end{itemize}

\section{Structure of the thesis proposal}

This proposal includes four sections and each will convey the related works and output when doing the thesis.

\begin{table}[H]
	\centering
	\begin{tabular}{ |c|p{13.5cm}| } 
		\hline
		Chapter          & Content                                                                                       \\
		\hline
		1                & A brief introduction about plan and objectives of thesis                                      \\
		\hline
		2                & Related works that had been done and how they help us in doing the thesis                    \\
		\hline
		\multirow{2}*{3} & Introduction of theoretical background as foundation knowledge that are applied in the thesis \\
		\hline
		4                & Solution and design approach for problem statement of the thesis                             \\
		\hline
		5                & Result and evaluation for the thesis                                               \\
		\hline                                                               \\
		6                & Summary of the thesis proposal                                                               \\
		\hline
	\end{tabular}
	\caption{Structure of the thesis proposal}
\end{table}
\chapter{Related Work}
Lorem Ipsum


\chapter{Theoretical Background}

\section{Convolution Neural Network - CNN}

Convolution Neural Networks are a particular class of Neural Networks \cite{masood2018real}. They are made up of neurons that have learnable weights and biases. Each neuron receives some inputs, performs a dot product, and optionally follows it with a non-linearity. CNN mainly consists of Convolution Layers, Pooling Layers, Activation Layers, and Fully Connected Layers. ConvNet architectures make the explicit assumption that the inputs are images, which allows us to encode specific properties into the architecture. These then make the forward function more efficient to implement and vastly reduce the number of parameters in the network. Some of the primary uses of CNN can be mentioned as image classification, object detection, semantic segmentation, face recognition, ...

%Insert picture about CNN https://scholarworks.iupui.edu/bitstream/handle/1805/24768/FINAL%20Prasham_Shah_Thesis%20.pdf?sequence=1&isAllowed=y
\begin{figure}[H]
	\centering
	\includegraphics[width=\textwidth]{img/Chap3/Cover.png}
	\caption{Convolution Neural Network}
	\label{fig:Chap3-OverviewTheCNN}
\end{figure}

The figure \ref{fig:Chap3-OverviewTheCNN} above shows an example of a convolution neural network, which is taking an image as input and then extracting features from it through various layers and then finally predicting the class of the object in the given image.

\subsection{Architecture}

Convolution Neural Networks have a different architecture with regular Neural Networks, and we can see this difference in figure \ref{fig:Chap3-DiffArchCNN_NNN} below. Regular Neural Networks transform an input through a series of hidden layers. Every layer comprises a set of neurons, where each layer is fully connected to all neurons in the previous layers. Finally, a last fully-connected output layer represents the predictions with CNN architecture. First of all, the layers are organized into three dimensions: width, height, and depth. Further, the neurons in one layer do not connect to all neurons in the next layer but only to a small region. Lastly, the system will reduce the final output to a single vector of probability scores, organized along the depth dimension.

%FIXME: Insert image of diff architecture : https://www.freecodecamp.org/news/an-intuitive-guide-to-convolutional-neural-networks-260c2de0a050/
\begin{figure}[H]
	\centering
	\includegraphics[width=\textwidth]{img/Chap3/DiffArchCNN-ANN}
	\caption{Different between Normal Neural Network and Convolution Neural Network}
	\label{fig:Chap3-DiffArchCNN_NNN}
\end{figure}

%FIXME: Insert image of CNN arc: https://www.freecodecamp.org/news/an-intuitive-guide-to-convolutional-neural-networks-260c2de0a050/
\begin{figure}[H]
	\centering
	\includegraphics[width=\textwidth]{img/Chap3/CNN-Arch}
	\caption{Convolution Neural Network Architecture}
	\label{fig:Chap3-CNN_Arch}
\end{figure}

As we can see in figure \ref{fig:Chap3-CNN_Arch}, CNN can be divided into two parts:
\begin{enumerate}
	\item The hidden layers/ Feature extraction part\\
	In this part, the network will perform a series of convolutions and pooling operations while the features are detected. Imagine you have a picture of a zebra, these are the parts where the
	network would recognize: its stripes, two ears, and four legs.
	\item The Classification part\\
	The fully connected layers serve as a classifier on top of their extracted features. By using a provided algorithm, they will assign a probability for the objects on the image.
\end{enumerate}

\subsection{Feature extraction part}
\subsubsection{Convolutional Layer}

The convolution layer is the core building block of a Convolutional Network that does most of the computational heavy lifting. A convolution is executed by sliding the filter over the input. At every location, matrix multiplication is performed and it sums the result onto the feature map. This extracting features from images happen throughout the CNN's convolutional layers. This process is illustrated in figure \ref{fig:Chap3-CNN_Layer}

% FIXME: https://scholarworks.iupui.edu/bitstream/handle/1805/24768/FINAL%20Prasham_Shah_Thesis%20.pdf?sequence=1&isAllowed=y
\begin{figure}[H]
	\centering
	\includegraphics[width=\textwidth]{img/Chap3/ConvLayer}
	\caption{Convolution Neural Network Layer}
	\label{fig:Chap3-CNN_Layer}
\end{figure}

When the feature map is made, we can pass each value in the feature map through a non-linearity function, such as ReLU, sigmoid before it becomes the input of the next convolution layer.

Because the size of the feature map is always smaller than the input, we have to do something to prevent our feature map from shrinking. This is where we use padding (\ref{fig:Chap3-CNN_Padding}). A layer of zero-value pixels is added to surround the input with zeros so that our feature map will not shrink. In addition to keeping the spatial size constant after performing convolution, padding also improves performance, ensures the Kernel and strides size will fit in the input.

% FIXME: => Need picture
\begin{figure}[H]
	\centering
	\includegraphics[width=\textwidth]{img/Chap3/CNN_Padding}
	\caption{Using padding for strike one in Convolution Layer}
	\label{fig:Chap3-CNN_Padding}
\end{figure}
\subsubsection{Pooling Layers}

After a convolution layer, it is common to add a pooling layer in between CNN layers. The function of Pooling is to continuously reduce the dimensionality to reduce the number of parameters and computation in the network. This action shortens the training time and controls overfitting.

There are two main types of Pooling Layers in a CNN: Max Pooling and Average Pooling. The functionality of these two types of layers is demonstrated in figure \ref{fig:Chap3-CNN_Pooling}. Max Pooling restores the maximum value from the picture segment covered by the Kernel. Average Pooling converts the average values from the bit of the picture surrounded by the Kernel.

\begin{figure}[H]
	\centering
	\includegraphics[width=\textwidth]{img/Chap3/Pooling}
	\caption{Max Pooling and Average Pooling}
	\label{fig:Chap3-CNN_Pooling}
\end{figure}

% FIXME: Insert image about max pooling and average pooling

\subsubsection{Activation Layers}

In general, Neural networks and CNNs rely on a non-linear "trigger" function to signal distinct identification of likely features on each hidden layer. CNN may use a variety of specific functions (figure \ref{fig:Chap3-CNN_ActiveFunction}), such as rectified linear units (ReLUs) and continuous trigger (non-linear) functions—to efficiently implement this non-linear triggering.

\begin{figure}[H]
	\centering
	\includegraphics[width=\textwidth]{img/Chap3/ActiveFunction}
	\caption{ Some Active Function common used in CNN }
	\label{fig:Chap3-CNN_ActiveFunction}
\end{figure}
% FIXME: Insert picture of some function like ReLU, tanh ....
\subsection{Classification part}
\subsubsection{Fully connected layers}

The last layers of a CNN are fully connected. Neurons in a fully connected layer have complete connections to all the activations in the previous layer. This part is, in principle, the same as a regular Neural Network.

Figure \ref{fig:Chap3-FC} illustrates the way of input value stream into the fully connected layer. Because these fully connected layers can only accept one-dimensional data, we need to convert our 3D data to 1D data. After passing through some FC, we will get the data classification result.

\begin{figure}[H]
	\centering
	\includegraphics[width=\textwidth]{img/Chap3/FC}
	\caption{ Fully connected Layer}
	\label{fig:Chap3-FC}
\end{figure}
% FIXME: Insert picture ...
\section{Media Pipe}\label{sec:MediaPipe}
% FIXME: Lấy từ eureka bỏ vào
\subsection{Introduction to Media Pipe Hands}
MediaPipe Hands (\ref{fig:Chap3-MediaPipe}) is a high-resolution tracking system for hands and fingers \cite{zhang2020mediapipe}. It uses machine learning to infer 21 3D hand landmarks from a single frame. This solution delivers real-time performance on a cell phone and even scales to many hands, whereas current state-of-the-art systems rely primarily on powerful desktop environments for inference.

\begin{figure}[H]
	\centering
	\includegraphics[width=0.8\textwidth]{img/Chap3/Media Pipe}
	\caption{ Media Pipe real time tracking 3D hand landmarks}
	\label{fig:Chap3-MediaPipe}
\end{figure}

MediaPipe Hands makes use of a machine learning pipeline that consists of several models that work together: A palm detection model, which acts on the entire image, will return an orientated hand bounding box. A hand landmark model that returns high-fidelity 3D hand key points from the cropped image region determined by the palm detector.

However, providing the hand landmark model with a correctly cropped hand image minimizes the requirement for data augmentation drastically (such as rotations, translations, and scaling) and instead, allows the network to focus on coordinate prediction accuracy. Furthermore, in this ML pipeline, crops can be created based on the hand landmarks recognized in the previous frame, and palm detection is only used to localize the hand when the landmark model can no longer detect its presence.
\subsection{Palm detection model}
The Media Pipe team provides the palm detection model to detect initial hand locations and distinguish whether the hand recognized is left or right, which is very useful as each sign goes along with a different side will result in different meanings. They created a single-shot detector model, comparable to the face detection model in MediaPipe Face Mesh \cite{MediaPipeFaceMesh}, tailored for mobile real-time applications. Hand detection is difficult: our model must detect occluded and self-occluded hands and work across many hand sizes with a significant scale span relative to the image frame.

According to their statements, the methods they used to address the above challenges vary in many strategies. First, instead of training a hand detector, they trained a palm detector because estimating bounding boxes of inflexible objects like palms and fists was much easier than recognizing hands with articulated fingers. Furthermore, the non-maximum suppression method performs effectively even in two-hand self-occlusion situations such as handshakes because palms are small objects. Furthermore, palms can be simulated using square bounding boxes (anchors in ML language) that ignore other aspect ratios, reducing 3-5 anchors. Second, even for tiny objects, an encoder-decoder feature extractor is used for more extensive picture context awareness (similar to the Retina Net approach). Finally, the significant scale variance limits focus loss during training to support many anchors.

Using the strategies described above gives an average precision of 95.7 percent in palm detection. With no decoder and a regular cross-entropy loss, the baseline is just 86.22 percent.

\subsection{Hand landmark model}
Following palm detection over the entire image, our next hand landmark model uses regression to accomplish exact key point localization of 21 3D hand-knuckle coordinates (see figure \ref{fig:Chap3-HandLandMark}) within the detected hand regions, i.e., direct, coordinate prediction. Even with partially visible hands and self-occlusions, the model develops a consistent internal hand posture representation.
\begin{figure}[H]
	\centering
	\includegraphics[width=\textwidth]{img/Chap3/HandLandMark}
	\caption{ 21 Hand Landmarks }
	\label{fig:Chap3-HandLandMark}
\end{figure}


\section{Distance Matrix}

A distance matrix \cite{DistanceMatrix} is a table that shows the distance between pairs of objects. For example, in the figure \ref{fig:Chap3-DM}., we can see the length between A and B is 16, B and C is 37, and so on. The diagonal of the table is the distance to the object from itself, so the value, as we can see, is 0. Distance matrices are sometimes called dissimilarity matrices.

\begin{figure}[H]
	\centering
	\includegraphics[width=0.6\textwidth]{img/Chap3/DM}
	\caption{ Distance Matrix }
	\label{fig:Chap3-DM}
\end{figure}

% FIXME: Insert picture of distance matrix

\subsection{Create Distance Matrix}

A distance matrix is computed from a raw data table. In the example below (figure \ref{fig:Chap3-DM_Formula}), we can use high school math (Pythagoras) to work out the distance between A and B.

% FIXME: Chèn công thức vào đây
\begin{figure}[H]
	\centering
	\includegraphics[width=0.7\textwidth]{img/Chap3/DM_Formula}
	\caption{ Calculating distance between A and B}
	\label{fig:Chap3-DM_Formula}
\end{figure}

We can use the same formula with more than two variables, known as the Euclidean distance. As a result, we have the distance matrix represented like figure \ref{fig:Chap3-DM-Raw}.

% FIXME: chèn bảng kết quả vào
\begin{figure}[H]
	\centering
	\includegraphics[width=0.7\textwidth]{img/Chap3/DM-Raw}
	\caption{ The Distance Matrix is constructed from Raw Data }
	\label{fig:Chap3-DM-Raw}
\end{figure}

\section{Beam search and Connectionist Temporal Classification}
% CheckList:
%   [x] BeamSearch
%   [x] CTC recap
%   [x] Combination
%   [x] Pseudo code
\subsection{Connectionist Temporal Classification}

Connectionist Temporal Classification (CTC) \cite{hannun2017sequence} is a type of Neural Network output helpful in tackling sequence problems like handwriting (figure \ref{fig:Chap3-Overview-CTC}) and speech recognition where the timing varies. Using CTC ensures that one does not need an aligned dataset, which makes the training process
more straightforward.

% FIXME: Insert about CTC in speech recognition
\begin{figure}[H]
	\centering
	\includegraphics[width=\textwidth]{img/Chap3/Overview-CTC}
	\caption{ Overview of a Neural Network for handwriting recognition }
	\label{fig:Chap3-Overview-CTC}
\end{figure}

\subsection{Why we want to use CTC}

In the context of handwritten recognition, we could create a dataset with images of text-lines, and then specify for each horizontal position of the image the corresponding character as shown in figure \ref{fig:Chap3-Annottion-image-CTC} Then, we could train a model to output a character-score for each horizontal position. However, there are two problems with this solution.

\begin{itemize}
	\item It takes much time, and annotating the dataset at the character level is tiresome.
	\item What if the character takes up more than one time-step ?. We could get "tooo" because the "o" is a wide-character as shown in figure \ref{fig:Chap3-Annottion-image-CTC}. We must remove all duplicate characters like "t" and "o".
\end{itemize}

% FIXME: Insert image ...
\begin{figure}[H]
	\centering
	\includegraphics[width=0.6\textwidth]{img/Chap3/Annotation-image-CTC}
	\caption{ Annotation for each horizontal position of the image }
	\label{fig:Chap3-Annottion-image-CTC}
\end{figure}

CTC can solve both problems for us:
\begin{itemize}
	\item We can ignore both the position and width of the character in the image and only requires the text that occurs in the picture.
	\item Using decode techniques, we can directly get the result of the network, and no further post-processing of the recognized text is needed.
\end{itemize}

\subsection{Beam Search with CTC decoder}
CTC has more than the Decoding phase, it can have the Encoding, Loss calculation, but we don't need it in the thesis scope anymore. So, here, we only mention to CTC decoder, but in the way, it combines with Beam Search \cite{scheidl2018word}. Because CTC in decoding context can connect with another algorithm like best-path decoding, ...

\subsubsection{Beam search}

In computer science, beam search \cite{BeamSearch} is a heuristic search algorithm that explores a graph by expanding the most promising node in a limited set. Beam search is an optimization of best-first search the reduces its memory requirements. Best-first search is a graph search that orders all partial solutions (states) according to some heuristic. But in beam search, only a predetermined number of the best partial solutions are kept as candidates. Pseudocode for the basic version of beam-search is shown in figure \ref{fig:Chap3-Basic-Version-BeamSearch}

\begin{figure}[H]
	\centering
	\includegraphics[width=0.8\textwidth]{img/Chap3/Basic-Version-BeamSearch}
	\caption{ Basic version of Beam Search }
	\label{fig:Chap3-Basic-Version-BeamSearch}
\end{figure}

% FIXME: Insert image about pseudo-code for beam-search

The beam search algorithm will be implemented through the following steps, with two parameters will be included: output matrix and beam width (BW), which specifies the number of beams to keep. First, the beam list and corresponding score are initialized (lines 1 and 2). After that, from 3-15, the algorithm will loop over all time-steps of the matrix output. At this point, only the best scoring beams (equal BW) from the previous time-step are kept (line 4). For each beam, we calculate the score and get a result (line 8); we will cover this step in more details later. Further, each beam is extended by all possible characters from the alphabet (line 10), and again, a score is calculated (line 11). After the last time-step, the best beams are returned (line 16).

% FIXME: Insert image about beam search

\begin{figure}[H]
	\centering
	\includegraphics[width=\textwidth]{img/Chap3/BeamSearchTree}
	\caption{ NN output and tree of beams with alphabet = {"a", "b"} and BW = 2 }
	\label{fig:Chap3-BSTree}
\end{figure}

As we can see, in figure \ref{fig:Chap3-BSTree}, the output matrices are decoded, and the tree of beams is shown. Beam search algorithm extended as possible and keep exactly BW candidates. Finally, we finished the last iteration, and the final step of the algorithm is to return the beam with the highest score, which is "a" in this example.

\subsubsection{Calculating the score}

As discussed above, in this part, we will talk about how to score the beam. We will split the beam-score into the score of paths ending with a blank(e.g. 'aa-') and paths ending with non-blank (e.g. 'aaa').

\begin{itemize}
	\item We denote the probability of all paths ending with a blank and corresponding to a beam b at time-step t
	      by $ P_{b}(b,t) $ and by $ P_{nb}(b,t) $ for the non-blank case.
	\item The probability $P_{tot}(b,t)$ of a beam b at time-step t is simply the sum of $P_b$ and $P_{nb}$, for example:
	      $P_{tot}(b,t) = P_b(b,t) + P_{nb}(b,t)$
\end{itemize}

\begin{figure}[H]
	\centering
	\includegraphics[width=0.7\textwidth]{img/Chap3/CTC_Scoring}
	\caption{ The effect of appending a character to paths ending with blank and non-blank }
	\label{fig:Chap3-CTC_Scoring}
\end{figure}

In figure \ref{fig:Chap3-CTC_Scoring}, we will see what happens when we extend a path. Three main cases we can mention
is:
\begin{itemize}
	\item Extended by blank ('a' + '-' = 'a-')
	\item Extended by repeating last character ( 'aa' + 'a' = 'aaa' or 'aa-' + 'a' = 'aa-a')
	\item Extended by some other character ('aa' + 'b' = 'aab')
\end{itemize}

% FIXME: Viết lại các công thức bên dưới
And when we collapse the extended paths, two results we will get and some cases we needed to handle:
\begin{itemize}
	\item The unchanged (copied) beam ('a' $ \rightarrow $ 'a'):
	      \begin{itemize}
		      \item To copy a beam, we can extend corresponding paths by a blank and get
		            paths ending with a blank: $ P_b (n, t) += P_{tot}(b, t-1)*mat(blank, t) $
		      \item Besides, with the non-blank ending paths case, if we extend it by the last
		            character (the beam is not empty): $ P_{nb}(b,t) += P_{nb}(b,t-1)*mat(b[-1],t) $
		            with -1 indexes the last character in the beam
	      \end{itemize}
	\item An extended beam ('a' $\rightarrow$ 'aa' or 'ab'):
	      \begin{itemize}
		      \item To extend a beam. With the last character is different from the character we need
		            to extend, then there is no need for separating blanks ('-') in the paths:
		            $ P_{nb}(b+c,t) += P_{tot}(b,t-1)*mat(c,t) $
		      \item Or the last character of beam is repeated, we must ensure that the paths
		            end with a blank: $ P_{nb}(b+c,t) += P_b(b,t-1)*mat(c,t) $
		      \item We don't need to care about $P_b(b+c,t)$ because we added a non-blank character
	      \end{itemize}
\end{itemize}

\subsubsection{Putting it all together}

The CTC beam search algorithm is depicted in the figure \ref{fig:Chap3-BS_CTC}. It is similar to the basic version that was previously shown. It does, however, contain the code for scoring the beams: copied beams (lines 7–10) and extended beams (lines 15–19). Finally, in order to find the best scoring beams, the program rates them using the Ptot (line 4) and then selects the best beams (BW).

\begin{figure}[H]
	\centering
	\includegraphics[width=0.8\textwidth]{img/Chap3/BS_CTC}
	\caption{ CTC beam search }
	\label{fig:Chap3-BS_CTC}
\end{figure}

\newpage
\section{Technology}
%FIX-ME: Chém gió thêm ra phần này, sử dụng ngôn ngữ gì bla bla 
% Ứng dụng được viết bằng react-native. Sử dụng hệ thống authenticate của firebase
Overall, the product app has been developed with technologies such as Java for Android apps and Firebase system authentication. Furthermore, to keep the application as light as possible, we use only native components and no external UI libraries.

% \subsection{React Native}

% \begin{figure}[H]
% 	\centering
% 	\includegraphics[width=0.7\textwidth]{img/technology/ReactNative.png}
% 	\caption{React Native Logo}
% 	\label{fig:ReactNativeLogo}
% \end{figure}

% Traditional mobile app development necessitates knowledge of two distinct platforms (and programming languages): Android and iOS. However, using React Native\cite{ReactNative}, developers can create hybrid apps that operate on both platforms. The advantages of applying React Native in the app development process contain the following points.

% \begin{itemize}
% 	\item \textbf{Cross-Platform:} One of the significant advantages of React Native is that we can write the same code for both the Android and iOS ecosystems simultaneously, with just minor changes for each platform.
% 	\item \textbf{One programming language:} There is no need to be familiar with the languages used for platform-specific application development because React Native employs JavaScript, which is currently one of the most popular programming languages\cite{10MostPopularProgrammingLang}. To be more specific, in this project, we use an advanced version of JavaScript, which is known as TypeScript\cite{TypeScript}, and we will discuss it in the later section.
% 	\item \textbf{Performance:} Because both platforms use the same code, React Native allows for the rapid creation of mobile applications. It also has a hot reloading functionality that ensures that modest changes to the program are shown to the developer right away.
% \end{itemize}

% Therefore, using this React Native framework benefits us in developing the same app for iOS devices from Apple Inc. According to the plan that we came up with before, when the product app works acceptably on Android devices, we will expand it and make it work on those iOS devices without massive effort when translating an app from one operating system to another.

% \subsection{TypeScript}

% \begin{figure}[H]
% 	\centering
% 	\includegraphics[width=0.2\textwidth]{img/technology/JavaScript.png}
% 	\caption{JavaScript Logo}
% 	\label{fig:JavaScriptLogo}
% \end{figure}

% To know about TypeScript, first, we need to know JavaScript. JavaScript is a solid client-side programming language that is open-source. Its primary purpose is to enhance users' interaction with a web page. In other words, developers may make the website more dynamic and exciting by using this programming language. JavaScript is also widely used in the development of games and mobile apps. Excellent speed, cross-browser interoperability, and simple semantics are some of JavaScript's primary qualities, enabling a seamless development experience.

% \begin{figure}[H]
% 	\centering
% 	\includegraphics[width=0.7\textwidth]{img/technology/TypeScript.png}
% 	\caption{TypeScript Logo}
% 	\label{fig:TypeScriptLogo}
% \end{figure}

% On the other hand, TypeScript is a modern JavaScript programming language. It is a statically built language for writing concise JavaScript code. It can be used with Node.js or any browser that supports ECMAScript 3 or above. Static typing, classes, and an interface are all available in TypeScript. Adopting Typescript for a large JavaScript project can result in more robust software easily deployable with a typical JavaScript application.

% \begin{itemize}
% 	\item \textbf{Static typing:} JavaScript is tightly typed, so it does not know what kind of variable it is dealing with until it is practically created at runtime. JavaScript now has type support thanks to TypeScript.
% 	\item \textbf{Supports new ECMAScript:} TypeScript may trans-pile unique ECMAScript criteria to ECMAScript objectives of our choice. As a result, we can use lambda, modules, functions, the spread operator, de-structuring, and classes, which are all features of ES2015 and beyond.
% 	\item \textbf{Type inference:} The use of type inference in TypeScript makes typing more manageable and less confusing. Even if we do not use the interface, TypeScript can help us avoid mistakes that could cause runtime issues.
% 	\item \textbf{Interoperability:} TypeScript is inextricably linked to JavaScript. As a result, it has high interoperability capabilities; however, it requires some additional work to integrate with TypeScript JS libraries.
% 	\item \textbf{Null examination strict:} In JavaScript software programming, errors, like cannot read property 'x' of undefined, are common. We may avoid most of these mistakes because a variable unknown to the TypeScript compiler cannot be used.
% \end{itemize}

% \subsection{NativeBase}

% \begin{figure}[H]
% 	\centering
% 	\includegraphics[width=0.7\textwidth]{img/technology/nativebase-logo.png}
% 	\caption{NativeBase Logo}
% 	\label{fig:NativeBaseLogo}
% \end{figure}

% NativeBase\cite{NativeBase} is a component library for developers that want to create universal design systems. It is based on React Native and lets developers create apps for Android, iOS, and the web.

% \begin{itemize}
% 	\item \textbf{Highly themeable:} NativeBase's themeability is one of its most essential features. Make the app's theme and component styles as unique as we like, which means we can change the theme effortlessly.
%  \item \textbf{Rich component library:} As GeekyAnts stated, NativeBase has approximately 40 components to let developers create an app quickly and easily. It includes action sheets, menus, spinners, popovers, breadcrumbs, and valuable components.
%  \item \textbf{Out of the box accessibility:} By default, we can use all the provided components without changing them much. Moreover, the things they provide are designed with a high contrast ratio, making it more comfortable for the users when using the app.
% \end{itemize}

\subsection{Java}

\begin{figure}[H]
	\centering
	\includegraphics[width=0.7\textwidth]{img/technology/java.png}
	\caption{Java Logo}
	\label{fig:JavaLogo}
\end{figure}

James Gosling developed Java at Sun Microsystems, Inc. in 1995, later acquired by Oracle Corporation. It's a straightforward programming language. Java makes programming easy to write, compile, and debug. It aids in the development of reusable code and modular programs.

Java is an object-oriented programming language based on classes that are designed to have as few implementation dependencies as feasible. Furthermore, Java is a general-purpose programming language that allows developers to write code once and run it on any device that supports Java. Java programs are compiled into byte code that may be run on any Java Virtual Machine. Besides, Java has a syntax that is similar to C and C++.

\subsection{Firebase}

\begin{figure}[H]
	\centering
	\includegraphics[width=0.7\textwidth]{img/technology/firebase.png}
	\caption{FireBase Logo}
	\label{fig:FireBaseLogo}
\end{figure}

Firebase is a web and mobile app development platform with simple and powerful APIs that don't require a server or backend. Firebase is a cloud-based platform. Also present is Google's server system. Its main goal is to make database operations simpler for users so they can program apps more easily. The real-time database service allows users to store and synchronize data. This is a completely cloud-based service. If the device is offline, it will use up its memory before automatically syncing with the server once it is online.

This feature largely comprises of backend services that help developers construct and manage their apps more effectively. The following services are included in this feature:

\begin{itemize}
	\item \textbf{Realtime database:} The Firebase Realtime Database is a cloud-based NoSQL database that processes data at millisecond speed. In the simplest sense, it can be thought of as a large JSON file.
 \item \textbf{Cloud firestore:} The Cloud Firestore is a NoSQL document database that allows users to store, sync, and query data from anywhere in the world using the app. It stores information in the form of documents, which are objects. It stores any data type, including text, binary data, and even JSON trees, using a key-value pair.
 \item \textbf{Authentication:} Using UI libraries and SDKs, the Firebase Authentication service makes it simple to authenticate users in the app. It reduces the time and effort required to develop and maintain the user authentication service. It even handles processes like account mergers, which can take a long time if done manually.
 \item \textbf{Remote configuration:} The remote configuration service speeds up the distribution of updates to users. Changes could range from updating UI components to altering application functionality. These are frequently used to deliver limited-time offers and content to a mobile application.
 \item \textbf{Hosting:} Firebase provides fast and secure application hosting. It can host both static and dynamic websites, as well as microservices. With a single command, it can host an application.
\end{itemize}

\chapter{Design and Solution}
\section{Gathering Data}
% TODO: Thu thập dữ liệu và phân loại như thế nào -> Lấy từ web ..., các video thủ ngữ trên zootube
% TODO: Tiến hành tổng hợp và phân loại -> tổng hợp thành bảng, phần loại từ thành các pattern, location, direction để chuẩn bị cho mấy mục dưới
% TODO: Chèn hình về cái sheet vào



Before we can design a system that can translate sign language, we must first understand what a sign language word is and where to collect the data. After observing the sign language on the websites \url{https://tudienngonngukyhieu.com/} and \url{https://qipedc.moet.gov.vn/}, we noticed that some of the words have the same pattern. Furthermore, the sign's meaning is determined by its orientation and location. Furthermore, some of the terms require hand or finger movements to represent the meaning. As a result, using those four factors, we can convert a word from sign language to Vietnamese.

However, we must first collect sign language data for the thesis's model training phase. Fortunately, those two websites contain a sufficient number of sign language words. We also learned a few words from YouTube videos taught by Mrs. Le Thi Thu Xuong \cite{yt:LeThiThuXuong} and channel CDS, Central Deaf Services in Dang Nang, Vietnam \cite{yt:CDS}. As a result, we can independently train and test our system.

We gathered a large number of words from those two websites and YouTube channel to prepare the data. Then we label and divide it into many elements, which we will discuss later in the Section \ref{sec:handstate}. For the data we collected, we created a Google Sheet. We have prepared many labeled words in this file (see Figure \ref{fig:Chap4-Label-Word}). In addition, we have a sheet for many different hand shapes (see Figure \ref{fig:Chap4-Sheet-Pattern}), which helps us classify hand patterns.

\begin{figure}[H]
	\centering
	\includegraphics[width=0.9\textwidth]{img/Chap4/Label-Word.jpg}
	\caption{Google sheet about words labeled}
	\label{fig:Chap4-Label-Word}
\end{figure}

\begin{figure}[H]
	\centering
	\includegraphics[width=0.9\textwidth]{img/Chap4/Sheet-Pattern.png}
	\caption{Google sheet about hand shapes can be recognized}
	\label{fig:Chap4-Sheet-Pattern}
\end{figure}



\section{System Structure}

Overall, the system is composed of three hardware modules: a camera module, the user's smartphone, and the server. The server, which will be the focus of this thesis, is one of those modules that handles the most complex work.

Our artificial intelligence system for sign language translation consists of six major modules: hand pattern recognition, direction determination, location detection, action detection, word decoder, and text to speech (Figure \ref{fig:Chap4-OverviewOfTheSystemModules-Old}). To begin, the system continuously captures the motion of the hand, processes it with the hand landmark model, and then stores it in those modules. Each of them plays a distinct role. The word decoder module will take the output data and produce the corresponding outcome after combining the results of the first four modules (hand pattern, direction, location, and action detection). The result will then appear on the main screen, while the phone will speak out that word.

\begin{figure}[H]
	\centering
	\includegraphics[width=0.9\textwidth]{img/Chap4/OverviewOfTheSystemModules-Old.png}
	\caption{Overview of the old system structure}
	\label{fig:Chap4-OverviewOfTheSystemModules-Old}
\end{figure}

Those six major modules mentioned above are the ones we had planned for at the start of the thesis. Despite going through most of the modules, we found it difficult to build the action detection module during the implementation period. It is problematic because of the demands it places on the smartphone and server.

There are some sign language words that contain a lot of moving patterns. The first method is to send the entire video captured by the camera to the server for processing. This method, on the other hand, necessitates a strong connection between the camera module, smartphone, and server and places strain on the physical devices (the camera and smartphone); as a result, those devices will become hot and may be damaged in some way. Another option is to increase the frame rate to get the action, but this can put the devices under strain. Furthermore, we must have an algorithm that is constantly processing and detecting movements, which we admit is difficult to achieve.
To solve the problem, we had to deprecate that module and change our method for obtaining the correct Vietnamese word. Instead of combining the four modules, including the action detection module, there are now only three remaining: pattern, direction, and location. Furthermore, in the word decoder module, we use the beam search heuristic search algorithm, which uses the results of the three modules to look up the word in the database and return it to us. Section \ref{sec:DetailImplementation} will go over each module's role and how it works.

\begin{figure}[H]
	\centering
	\includegraphics[width=0.9\textwidth]{img/Chap4/OverviewOfTheSystemModules-New.png}
	\caption{Overview of the new system structure}
	\label{fig:Chap4-OverviewOfTheSystemModules-New}
\end{figure}

\section{Detail Implementation}\label{sec:DetailImplementation}

\subsection{Hand pattern recognition}

The first and most fundamental module of this system is hand pattern recognition. When a person with a disability performs sign language, their hands move in a variety of ways, with their hands spread out, clenched, or their fingers pointing out at something. As a result, the role of this module is to recognize the hand pattern. The system can then provide the final result by combining the outcome with other modules.

This module makes use of the hand landmark model's output, which has a matrix size of 21. We get a new matrix representing the distance between those 21 coordinates after calculating all of the values in that matrix. As seen in Figure \ref{fig:Chap4-OverviewOfTheSystemModules-New}, using the distance matrix as the input of CNN with the designed structure (see Figure \ref{fig:Chap4-StructureOfConvolutionalNeuralNetwork}) will tell us the pattern of the hand at the time it is captured.

\begin{figure}[H]
	\centering
	\includegraphics[width=\textwidth]{img/Chap4/Hand-Pattern-Reg-Model.png}
	\caption{Hand pattern recognition pipe line}
	\label{fig:Chap4-StructureOfConvolutionalNeuralNetwork}
\end{figure}

\subsection{Direction determination}

The hand has four directions, which are right, left, up, down, front, and back. The pattern of each hand, when combined with different directions, yields a different meaning. For example, the pattern that points at someone means the word "you," whereas the pattern that points at ourselves means the word "I." (see Figure \ref{fig:Chap4-WordYouInSignLanguage} and Figure \ref{fig:Chap4-WordIInSignLanguage}).

\begin{figure}[H]
	\centering
	\includegraphics[width=0.6\textwidth]{img/Chap4/WordYouInSignLanguage.png}
	\caption{Word "You" (bạn) in sign language}
	\label{fig:Chap4-WordYouInSignLanguage}
\end{figure}

\begin{figure}[H]
	\centering
	\includegraphics[width=0.6\textwidth]{img/Chap4/WordIInSignLanguage.png}
	\caption{Word "I" (tôi) in sign language}
	\label{fig:Chap4-WordIInSignLanguage}
\end{figure}

We use the hand landmark model provided by MediaPipe to determine the direction of the hand (see Section \ref{sec:MediaPipe}). The idea here is to calculate the distance between the tip of the index finger and the wrist, which is known as a \textbf{vector(0, 8)}, and then project it to the axes Ox, Oy, and Oz, in that order. Following that, we compare those coordinates to the others. Finally, the one of immense value will tell us which axis the hand is on; additionally, we will know which direction the hand is by projecting the direction from the wrist to the tip of the index finger on that corresponding axis.

\begin{figure}[H]
	\centering
	\includegraphics[width=\textwidth]{img/Chap4/DirectionSteps.png}
	\caption{Steps to detect the direction of the hand}
	\label{fig:Chap4-DirectionSteps}
\end{figure}

For instance, a hand is pointing in the left direction. When projected on the axis Ox, the distance value will be the largest of the three projected values. Then, compute the vector drawn from the wrist to the tip of the index finger; we'll know the hand's direction.

\begin{figure}[H]
	\centering
	\includegraphics[width=0.6\textwidth]{img/Chap4/vector0-8-forwardLeft.png}
	\caption{Vector(0, 8) represent the hand pointing toward the left}
	\label{fig:Chap4-vector0-8-forwardLeft}
\end{figure}

\subsection{Location detection}\label{sec:locationDetection}

Hand placement differs; is the hand on the brow, mouth, or chest level, and so on. Each hand pattern associated with each location will yield a different set of words. Nonetheless, with only one camera and a view from above, it is difficult for the AI to determine the coordinates of the hand (see Figure \ref{fig:Chap4-ViewFromCamera}). However, we devised some solutions to this problem.

% The first method we use to determine the location of the hand is zooming. In this solution, we will take images of hands and calculate their size in each frame to determine whether those hands are growing or shrinking. As a result, if those hands are smaller than before, they are getting further away from the camera, and their locations are somewhere near the chest or stomach. Otherwise, the hand is closer to the camera, at the level of the mouth, nose, or forehead.

\begin{figure}[H]
	\centering
	\includegraphics[width=0.6\textwidth]{img/Chap4/ViewFromCamera.png}
	\caption{View from the camera module}
	\label{fig:Chap4-ViewFromCamera}
\end{figure}

% Nonetheless, the above solution has a flaw: each man's hand is different in size, and the system does not know the correct hand position. Another option is to use a wide-angle camera that is positioned away from the brow. The camera can have a much wider field of view with this solution. Nonetheless, because we only had a standard-angle camera, we couldn't test this solution and confirm its suitability.

The first solution to detect the hand's location is using an ultrasonic sensor. In short, this sensor is an instrument that measures the distance to an object using ultrasonic sound waves (see Figure \ref{fig:Chap4-UltrasonicSensorFunction}). It works by emitting a sound wave with a frequency above the human hearing range. The sensor's transducer functions as a microphone, receiving and transmitting ultrasonic sound. The sensor measures the time between sending and receiving an ultrasonic pulse to determine the distance to a target.

\begin{figure}[H]
	\centering
	\includegraphics[width=0.7\textwidth]{img/Chap4/UltrasonicSensorFunction.png}
	\caption{Illustration of how the ultrasonic sensor works}
	\label{fig:Chap4-UltrasonicSensorFunction}
\end{figure}

\begin{wrapfigure}{r}{0.475\textwidth}
  \begin{center}
  	\includegraphics[width=0.3\textwidth]{img/Chap4/UltrasonicSensor.jpeg}
  \end{center}
	\caption{The ultrasonic sensor HY-SRF05}
  \label{fig:Chap4-UltrasonicSensorHYSRF05}
\end{wrapfigure}

In the thesis, we use the ultrasonic sensor HY-SRF05 (Figure \ref{fig:Chap4-UltrasonicSensorHYSRF05}), which is relatively inexpensive and meets our requirement for measuring the distance between the camera and the hand. According to the retailer, this sensor has a scanning range of up to 15 degrees. Furthermore, its scanning range is between 2 cm and 450 cm, with a relative error of around 0.3 cm. Furthermore, the most accurate measurement distance is less than 100 cm, which is more than enough to measure from the user's forehead to their waist.

However, the sensor method has yet to produce the desired result. The sensor can only detect one hand at a time, but what if sign language uses both hands at the same time? In this case, we cannot achieve the best possible results with only one sensor.

From there, we propose the following solution for measuring this distance. Replace the distance measurement method based on object size with one that uses a location sensor. It is similar to the zooming method, but it is superior in that it does not depend on the size of the hand, making it ideal for determining the distance for the problem we are attempting to solve.

This method is as follows. We will use the following formula:
\begin{center}
  $ distance\_predict = A*distance\_input + B*distance\_input + C $ 
\end{center}

In which $distance\_input$ will be the distance between two points (5,17) on the hand model taken from the MediaPipe model, and $distance\_predict$ will be the distance from the camera to the hand.

The trio of coefficients A, B, and C will be determined by interpolating the above polynomial with a pre-prepared data set. After having the above three coefficients, combined with calculating the distance between 2 points 5 and 17, we will quickly deduce the hand's distance. 

Figure \ref{fig:Chap4-LocationModule} shows the result we get

\begin{figure}[H]
	\centering
	\includegraphics[width=0.7\textwidth]{img/Chap4/Location_00.jpg}
	\caption{Result of location module}
	\label{fig:Chap4-LocationModule}
\end{figure}

As a result, after receiving the result from this location module, we placed it in the hand state. Similarly, in the Section that follows, we will discuss how that hand state will aid us in translating sign language into Vietnamese.

% TODO
  % New approach : Sử dựng phương pháp khác để có thể đo khoảng cách
  % [ ] Thêm hình ảnh 
  % [ ] Giải thích
  
  % Thay thế phương pháp đo khoảng cách bằng sóng âm bằng phương pháp đo khoảng cách dựa trên hình ảnh 
  % Công thức: ...
  % Sử dụng phép nội suy đa thức để suy ra giá trị của các tham số A,B,C. 
  
  % Chúng ta sẽ tính khoảng cách giữa 2 điểm 5 và 17 trên bàn tay, sau đó sử dụng kết quả vừa tính toán được và công thức đã nêu ở trên để suy ra khoảng cách thực tế của bàn tay

  % Sau khi khoảng cách đã được ước lượng, ta sẽ suy ra được vị trí hiện tại của bàn tay đang ở đâu trên cơ thể (đầu, ngực hay bụng)

\subsection{Word decoder}
% TODO:   Previous approach 

% [x] How to map word ?

% TODO:   New approach

% [x] Punish function

% [x] Using beam search

% [x] CTC decode

% [x] Flow

% [x] Expected result

% [...] Difficult and proposed solution

% [...] Dịch sang tiếng anh

% [...] Thêm các hình ảnh


% TODO: With previous approach

As previously discussed, there are significant technical challenges in implementing the action detection module. We conducted research and proposed a new model to address these issues. As a result, this change has an impact on the word decoder module, which requires some tweaking.

The previous model breaks down a word into four components: pattern, location, direction, and action. It will search the database for the corresponding word after receiving the outputs from the four modules. Figure \ref{fig:Chap4-MapWord} demonstrates how a four-factor input is mapped to the correct word in the database. We have the system use a basic searching algorithm to find the most appropriate word. If none are found, it will replace or deprecate some parts of the input before attempting to find another word. The application will display the appropriate word on the screen after decoding it.

\begin{figure}[H]
	\centering
	\includegraphics[width=\textwidth]{img/Chap4/MapWord.png}
	\caption{Map one to one data from four component with word in database and get result}
	\label{fig:Chap4-MapWord}
\end{figure}

\subsubsection{ Introduction to handstate }\label{sec:handstate}

The question that arises immediately following the deprecation of the action detection module is how we can find the correct word without that module. As a result, unlike the previous model, we propose a different model for a word that is not encoded into four factors. It only has three remaining elements: pattern, direction, and location. As a result, each combination of those three elements is referred to as a hand state (Figure \ref{fig:Chap4-HandState}), and a word can be decoded into a variety of hand states.

This concept of hand state stems from natural language processing research, in which a word is made up of many characters. As a result, a word is concatenated from several hand states in the thesis. Then, after going through the processing steps discussed later in this proposal, we will get the desired word.

\begin{figure}[H]
  \centering
  \includegraphics[width=0.6\textwidth]{img/Chap4/HandState.png}
  \caption{ Hand state which construct from pattern, location and direction}
  \label{fig:Chap4-HandState}
\end{figure}

\subsubsection{ Using beam search and CTC decode to map word}

After we have grasped the concept of hand state, we will proceed to the most important part of the model: converting the received hand states into words.      

\begin{figure}[H]
  \centering
  \includegraphics[width=\textwidth]{img/Chap4/Architechture.png}
  \caption{Architecture}
  \label{fig:Chap4-Architechture}
\end{figure}

The authors' model for this Section is shown in Figure \ref{fig:Chap4-Architechture}. A queue of hand states from the previous three components will be used as input. The queue length is set to 5 in our conventions, but this is not the final number because we need more calculations and experimentation to find the optimal queue length. This model is comprised of three steps:

\begin{enumerate}
  \item \textbf{Vectorization:} This step converts a queue of many hand states (Figure \ref{fig:Chap4-HandStateQueue}) into a matrix for beam search input.
  \item \textbf{Beam search:} In this step, we will use a beam search algorithm to determine which hand states are appropriate for the database input. Furthermore, we propose using the CTC decode model to eliminate incorrect or previously duplicated hand states, increasing the model's efficiency.
  \item \textbf{Map to the dictionary:} Finally, after going through the preceding two steps, we will obtain the most likely hand state from the initial queue. Our job is to match these hand states to the correct word in the database.
\end{enumerate}
      
\subsubsection{ Vectorization }
% TODO: Why we need punish
% TODO: how to perform -> Trình bày cách đánh giá như thế nào, cách trừ điểm và các phương châm đánh giá
% TODO: Sau khi punish dùng hàm softmax để chuyển các giá trị về dạng xác suất

We get a list of hand states when we get to this step. Because we need a matrix representing the correlation between the component outputs and the data in the database before entering the beam search module.

\begin{figure}[H]
  \centering
  \includegraphics[width=\textwidth]{img/Chap4/HandStateQueue.png}
  \caption{A queue of hand states to sign "Thank you"}
  \label{fig:Chap4-HandStateQueue}
\end{figure}

From this queue, we will cycle through each hand state, compare it to the available hand state database, and calculate the score based on the principles shown in Figure \ref{fig:Chap4-Vectorization}:

\begin{enumerate}
  \item If the word in the hand matches the word in the database, the score will be increased.
  \item Otherwise, if that hand state does not match any, the score will be reduced.
\end{enumerate}

\begin{figure}[H]
  \centering
  \includegraphics[width=\textwidth]{img/Chap4/Vectorization.png}
  \caption{ Vectorization }
  \label{fig:Chap4-Vectorization}
\end{figure}
% TODO: Change to Math

The first principle states that the higher the score, the more similar the hand state retrieved from the queue is to that in the database. For example, in the database, we have the following hand state:

$\begin{bmatrix}
  pattern \_ A & location \_ A & direction \_ A
\end{bmatrix}$
, and the hand state we get from the input is 
$\begin{bmatrix}
  pattern \_ A & location \_ A & direction \_ A
\end{bmatrix}$
then this hand state will be rated higher than the hand state 
$\begin{bmatrix}
  pattern \_ A & location \_ A & direction \_ B
\end{bmatrix}$
. And so forth. In turn, we will score the hand states taken from the queue.

The minus point is evaluated on the second point based on its matching pattern with the hand states in the database. When the system recognizes patterns from the hand pattern recognition module (via the vision approach), they are likely to be incorrectly detected. We devised a rule to address this issue and improve the accuracy of the results. The minus point will be lower for patterns that are difficult to detect than for simple ones. In short, the smaller the minus point, the more complex the pattern to be recognized.

% + Khi đánh giá các hand state, đối với trường hợp so trùng 2 pattern. Do các pattern này được nhận diện từ module hand pattern regconition (vision approach), do đó, sẽ có khả năng bị nhận diện bị sai, hoặc bị nhầm. Vì lẽ đó, để có thể đánh giá một cách chính xác và công bằng nhất có thể thì ở đây, đối với những pattern hay bị nhận diện sai, ta sẽ trừ điểm thấp và ngược lại, với những pattern đơn giản mà hệ thống lại nhận diện sai thì sẽ bị trừ điểm nhiều hơn.

Following the completion of the preceding evaluation and scoring steps, we will use a function to normalize the data (here, the authors use the softmax function \cite{SoftMax}) and return a set of probabilities for the hand states in the row. Wait. This set of probabilities will be used as input for the beam search step.

\subsubsection{ Using beamsearch with CTC decode }
% TODO: Trình bày cách sử dụng beamsearch để tìm các cặp bộ 3
% TODO: Image beamsearch (get from ppt)
% TODO: Example
% TODO: Áp dụng CTC để handle một số trường hợp
% TODO: Các khó khăn gặp phải và hướng giải quyết

The hand states in our queue have been converted to a MxN matrix after passing the vectorization step, where M is the length of the hand state's database and N is the length of the queue.

We will retrieve the most likely k-hand state from the database by using beam search (Figure \ref{fig:Chap4-BeamSearch}). We can imagine what happened later during the beam search based on the image below.

% Sau khi qua bước vectorizaion, các hand state trong hàng đợi của chúng ta
% đã được chuyển đổi thành một ma trận MxN với M là độ dài của cơ sở dữ liệu về
% các hand state và N là độ dài của hàng đợi.
      
% Bằng việc sử dụng beam search, từ sẽ thu được k hand state có khả năng nhất
% từ cơ sở dữ liệu. Từ hình ảnh bên dưới, ta có thể hình dung được những gì
% đã diễn ra sau trong quá trình thực hiện beam search

% FIXME: Insert image from ppt about matrix with beam search

\begin{figure}[H]
  \centering
  \includegraphics[width=\textwidth]{img/Chap4/BeamSearch.png}
  \caption{ Beam Search }
  \label{fig:Chap4-BeamSearch}
\end{figure}

However, as we can see, after the beam search step, we will get a sequence of hand states with a length equal to the length of the input queue. These hand states may contain duplicate hand states or incorrect hand states. As a result, we must use the CTC algorithm to remove the hand states from infection. We will set a threshold in the vectorization step to discard these hand states and treat them as a blank character for the incorrect hand states. Furthermore, we will get the desired result after beam search CTC decode.

% Tuy nhiên, có thể thấy được rằng, sau khi thực hiện xong bước beamsearch,
% ta sẽ thu được một dãy các hand state có độ dài tương ứng với độ dài của hàng đợi
% , các hand state này có thể bao gồm những hand state bị trùng nhau, hoặc cũng có thể
% là những hand state bị sai. Do đó, ta cần áp dụng thêm CTC decode để loại bỏ các hand state
% bị trùng này. Đối với những hand state bị nhận sai từ những module trước thì ở bước Vectorization,
% ta sẽ đặt một threshold để loại bỏ những hand state này và xem như hand state đó là một ký tự rỗng (" ")
% và cuối cùng, sau khi áp dụng CTC decode vào, ta sẽ thu được kết quả mong muốn

% FIXME: Insert image about the result (get from ppt)

      
      
    \subsubsection{ Map to dictionary }
      % TODO: Cách map như thế nào

      \begin{figure}[H]
        \centering
        \includegraphics[width=\textwidth]{img/Chap4/Result.png}
        \caption{ Map to dictionary and get word }
        \label{fig:Chap4-Result}
      \end{figure}

      % Sau khi nhận được một tập các hand state có khả năng nhất, việc còn lại là
      % chúng ta sẽ map vào cơ sở dữ liệu, như cách mà chúng ta đã làm trong mục ..., 
      % nhưng thay vì map với bộ 4 thành phần thì ở đây, ta sẽ map với input các hand state thu được
      % từ bước ... .
      % Và như thế, ta sẽ thu được từ vựng mà không cần phải dùng tới module action detection.

All that remains is for us to map to the database, as we did in the previous Section, but instead of mapping with a 4-component set, we will map with input retrieved from this previous step. Finally, without using the action detect module, we will obtain the word (Figure \ref{fig:Chap4-Result}).

\subsection{Text-to-speech}

Furthermore, because people do not always read the result from the phone's screen, the application can speak it out loud to make it easier for them to know the answer after the translation process. One method is to create a database of many sound files that have been tagged with the corresponding Vietnamese word. This approach, however, is inefficient because it necessitates a significant amount of effort to create the database. Every single word must be recorded and then mapped together.

Instead, we apply a well-known speech service from Google known as Text-to-Speech\cite{GG:Text-to-Speech}. Text-to-Speech, according to Google, converts text input into natural human speech audio data. Furthermore, this service supports a wide range of languages, including the one we require, Vietnamese. Using the API provided by that speech service, our application can speak up the result without requiring a large sound database on the user's phone.

This API is available for free. The cost of text-to-speech is determined by the number of characters sent to the service each month to be synthesized into audio. According to Google, "the first 1 million characters for WaveNet voices are free each month." Each month, the first 4 million characters for standard (non-WaveNet) voices are free. After the free tier, Text-to-Speech is charged per 1 million text characters processed. The thesis only requires the standard (non-WaveNet) plan, which includes 4 million characters for free each month and costs USD 4.00 for each additional 1 million characters. The number of characters we will use in the upcoming phases of building the application is negligible in comparison to the 4 million free characters. As a result, we decided to use the Google service to implement this Text-to-Speech module.

\subsection{The camera module}

After discussing the solutions and implementations of the system's soft modules, we must move on to the main one, which is considered the thesis's eyes, which is the camera module. This Section will go over the components of a camera module and show some images of a real one that we built.

Parts for the camera modules are easily available from any retailer that sells electrical components, robots, and Arduino kits. Furthermore, in this day and age of e-commerce, it is much easier to find and compare the components that we require online. The components needed to construct a camera module are listed below.

First, we require a camera component, which the ESP32-CAM is ideal for. It is inexpensive and simple to use, which makes it ideal for our thesis, which requires complex functions such as image tracking and recognition. Furthermore, it incorporates Wi-Fi and traditional Bluetooth, allowing us to send images to the user's smartphone for the next steps in sign language translation.

Second, we will require a converter adapter to assist us in sideloading the program into the camera module. The ultrasonic sensor mentioned in the Section \ref{sec:locationDetection} is the third component we require. It aids in location detection by informing the system of the distance between the hands and the camera module. Last but not least, this camera module requires a battery to power the entire module, and we believe that a volume of around 100 mAh is adequate.

Furthermore, all of the above components must be stored in a box. We designed that package on Tinkercad, an online 3D modeling program that runs in a web browser, using current 3D printing technology. After gathering all of the necessary components, we attempted to put them together and obtained the result shown below.

\begin{figure}[H]
	\centering
	\includegraphics[width=0.45\textwidth]{img/Chap5/Prototype_View_inside.png}
	\caption{The components inside the camera module prototype}
\end{figure}

\begin{figure}[H]
	\centering
	\begin{subfigure}[b]{0.45\textwidth}
		\centering
		\includegraphics[width=\textwidth]{img/Chap5/Prototype_View_above.png}
	\end{subfigure}
	\hfill
	\begin{subfigure}[b]{0.46\textwidth}
		\centering
		\includegraphics[width=\textwidth]{img/Chap5/Prototype_View_under.png}
	\end{subfigure}
	\caption{Views of the camera module prototype from the above and under}
\end{figure}

\begin{figure}[H]
	\centering
	\begin{subfigure}[b]{0.45\textwidth}
		\centering
		\includegraphics[width=\textwidth]{img/Chap5/Prototype_View_side_1.png}
	\end{subfigure}
	\hfill
	\begin{subfigure}[b]{0.45\textwidth}
		\centering
		\includegraphics[width=\textwidth]{img/Chap5/Prototype_View_side_2.png}
	\end{subfigure}
	\caption{Views of the camera module prototype from the sides}
\end{figure}

Nonetheless, the box's cover is difficult to insert due to the smallest number that a 3D printer can print. The hanger that assisted in hanging the box on the hat is not as flexible as we had hoped, so it needs to be redesigned.

\subsection{App design}

The application must meet user experience and user interface requirements. And, as part of the thesis research, we created a prototype for the application, which included one additional feature in addition to the main one. A sign language dictionary is one of the extra features. Table \ref{tab:4-dictionary} demonstrates this feature.

Before delving into the design of this application, it is important to note that they do not yet cover all of the screens required for the application. And they do not represent the final design that we have. However, when designing this prototype, we followed some conventions, such as rounding the corners and keeping the colors pale and not too bright, in order to make the users feel calm and at ease while using the application.

\begin{figure}[H]
	\centering
	\includegraphics[width=0.8\textwidth]{img/Chap5/Landing_page.png}
	\caption{The landing page of the application}
\end{figure}

\begin{figure}[H]
	\centering
	\includegraphics[height=0.8\textheight]{img/Chap5/Main_screen.png}
	\caption{The main screen of the application}
\end{figure}

% \begin{figure}[H]
% 	\centering
% 	\includegraphics[width=\textwidth]{img/Chap5/Dictionary.png}
% 	\caption{The main screen of the application}
% \end{figure}

\begin{figure}[H]
	\centering
	\includegraphics[width=\textwidth]{img/Chap5/Profile_screen.png}
	\caption{The profile screen}
\end{figure}

\begin{figure}[H]
	\centering
	\includegraphics[width=\textwidth]{img/Chap4/Dictionary.png}
	\caption{The dictionary screen}
\end{figure}


% Vẽ use-case
% [ ] Use-case 
%   + Xem kết quả từ được trả về sau khi dự đoán
%   + Đăng nhập/ Đăng xuất
%   + Đăng ký
%   + Xem profile
%   + Xem từ điển


% [ ] Use-case description     

\newpage
\subsection{Use case}

On the whole, Figure \ref{fig:Chap4-usecase} illustrates the project's overall use-case.

\begin{figure}[H]
	\centering
	\includegraphics[width=\textwidth]{img/Chap4/use-case.drawio.png}
	\caption{Use case diagram}
  \label{fig:Chap4-usecase}
\end{figure}

\subsection{Use case description}

\subsubsection{View translated result}
\begin{table}[H]
  \centering
  \begin{tabular}{ |l| p{11cm}|}
    \hline
    Use case ID & 1 \\ 
    \hline
    Use case name & View translated result \\ 
    \hline
        Description & The user is informed about the hand gesture he just made\\
        \hline
        Actor & User\\
        \hline
        Post-condition(s) & Successful sign language detection and return to the user\\
        \hline
        \multirow{4}*{Normal flow}  & 1. The user visits the main page\\
        						        & 2. The user signs the word\\
        					            & 3. Application that records actions and makes predictions\\
        					            & 4. The application displays the result after translated successfully\\
        \hline
        \multirow{3}*{Exception flow}   & Exception1: \\
                                            & 4a. The application cannot translate the sign language\\
                                            & 5a. The application shows no more information and ends\\
        \hline
  \end{tabular}
  \caption{Use case view translated result}
\end{table}



% Chỉnh sửa lại phần này, thêm alternative flow về việc chỉnh sửa profile
\subsubsection{View profile}
\begin{table}[H]
  \centering
  \begin{tabular}{ |l| p{11cm}|}
    \hline
    Use case ID & 2 \\ 
    \hline
    Use case name & View profile \\ 
    \hline
        Description & The user can view his personal information and edit the number of learned words and the number of new words learned in the day\\
        \hline
        Actor & User\\
        \hline
        Post-condition(s) & The user sees his information page\\
        \hline
        \multirow{3}*{Normal flow}  & 1. User visits the profile page \\
        						        & 2. The application connects to the system in order to obtain information about the user\\
        					            & 3. The application displays user data\\

        \hline
        Alternative flow & Null \\ 
        \hline
        Exception flow   & Null \\
        \hline
  \end{tabular}
  \caption{Use case view profile}
\end{table}

\subsubsection{Sign up}
\begin{table}[H]
  \centering
  \begin{tabular}{ |l| p{11cm}|}
    \hline
    Use case ID & 3 \\ 
    \hline
    Use case name & Sign up \\ 
    \hline
        Description & The user must first create a personal account to access the system\\
        \hline
        Actor & User\\
        \hline
        Pre-condition(s) & The device must be connected to the internet\\
        \hline
        Post-condition(s) & Successful account registration\\
        \hline
        \multirow{5}*{Normal flow}  & 1. The user taps on “Đăng ký tài khoản” button \\
        						        & 2. The application displays the account registration view\\
                            & 3. The user enters the necessary data\\
                            & 4. The user taps the "Đăng ký" button \\
                            & 5. The application redirect the user to the login screen\\
        \hline
        \multirow{4}*{Alternative flow}   & A. The user abandons the account creation process and wishes to return to the login screen\\
                                          & 3.1 The user taps the "Đăng nhập” button and the application jumps to step 5 \\
        \hline
        Exception flow   & If there is no internet connection, the application displays the message "Không có kết nối mạng, vui lòng thử lại sau." \\
        \hline
  \end{tabular}
  \caption{Use case sign up}
\end{table}

\subsubsection{Sign in}
\begin{table}[H]
  \centering
  \begin{tabular}{ |l| p{11cm}|}
    \hline
    Use case ID & 4 \\ 
    \hline
    Use case name & Sign in \\ 
    \hline
        Description & Allow the user to sign in to his account in order to use the application's services\\
        \hline
        Actor & User\\
        \hline
        Pre-condition(s) & The device must be connected to the internet\\
        \hline
        Post-condition(s) & The user successfully logs in.\\
        \hline
        \multirow{4}*{Normal flow}  & 1. The application displays the login page\\
        						        & 2. The user enters his email address and password in the appropriate fields.\\
        					            & 3. The user taps the "Đăng nhập" button\\
                              & 4. The application shows “Logged in successfully” \\ 
                              & 4. The application direct the user to the main screen\\ 
        \hline
        \multirow{12}* {Alternative flow}  & A. The user enters wrong email address\\
                                          & 4.1 The application shows "Email không tồn tại" \\ 
                                          & 4.2 Application goes back to step 2 \\ 
                                          & B. The user enters invalid email address
                                          4.1 The application shows "Email không đúng định dang" \\ 
                                          & 4.2 Application goes back to step 2 \\ 
                                          & C. The user enters wrong password\\
                                          & 4.1 The application shows “Mật khẩu không đúng” \\ 
                                          & 4.2 Application goes back to step 2\\
        \hline
        Exception flow   & If there is no internet connection, the application displays the message "Không có kết nối mạng, vui lòng thử lại sau." \\
        \hline
  \end{tabular}
  \caption{Use case sign in}
\end{table}


\subsubsection{Sign out}
\begin{table}[H]
  \centering
  \begin{tabular}{ |l| p{11cm}|}
    \hline
    Use case ID & 5 \\ 
    \hline
    Use case name & Sign out \\ 
    \hline
        Description & The user wants to sign out the current account\\
        \hline
        Actor & User\\
        \hline
        \multirow{2}*{Pre-condition(s)} & The device must be connected to the internet \\
                                        & The user logged in successfully \\ 
        \hline
        Post-condition(s) & The user signs out successfully\\
        \hline
        \multirow{2}*{Normal flow}  & 1. The user taps the "Đăng xuất" button \\
        						        & 2. The application returns to the login screen\\
        \hline
        Alternative flow  & No \\
        \hline
        Exception flow   & If there is no internet connection, the application displays the message "Không có kết nối mạng, vui lòng thử lại sau." \\
        \hline
  \end{tabular}
  \caption{Use case sign out}
\end{table}

\subsubsection{Look up word in dictionary}
\begin{table}[H]
  \centering
  \begin{tabular}{ |l| p{11cm}|}
    \hline
    Use case ID & 6 \\ 
    \hline
    Use case name & Look up word in dictionary \\ 
    \hline
        Description & The user needs to look up a word in the sign language dictionary\\
        \hline
        Actor & User\\
        \hline
        \multirow{1}*{Pre-condition(s)} & For the better result, the device should have internet connection \\
        \hline
        Post-condition(s) & The user successfully finds out how to sign the word\\
        \hline
        \multirow{2}*{Normal flow}  & 1. The user taps dictionary in the navbar\\
        						        & 2. The application directs the user to screen dictionary\\
        						        & 3. The user inputs the word and taps it on screen\\
        						        & 4. The application displays the word's corresponding information and includes an instructional video.\\
        \hline
        Alternative flow  & No \\
        \hline
        Exception flow   & If there is no internet connection, the application only shows the local information, and the video is note included. \\
        \hline
  \end{tabular}
  \caption{Use case look up word in dictionary}
  \label{tab:4-dictionary}
\end{table}

\subsubsection{Mark learned word}
\begin{table}[H]
  \centering
  \begin{tabular}{ |l| p{11cm}|}
    \hline
    Use case ID & 7 \\ 
    \hline
    Use case name & Mark learned word \\ 
    \hline
        Description & The user wishes to mark a word as learned in order to easily find it in the future.\\
        \hline
        Actor & User\\
        \hline
        \multirow{1}*{Pre-condition(s)} & Previously logged in \\ 
        \hline
        Post-condition(s) & The user marked the word, which increased the learning bar on the profile page\\
        \hline
        \multirow{2}*{Normal flow}  & 1. The user taps dictionary in the navbar\\
        						        & 2. The application directs the user to screen dictionary\\
        						        & 3. The user inputs the word and taps it on screen\\
        						        & 4. The application shows the corresponding information of the word, includes the instruction video\\
        						        & 5. User taps on the bookmark button on the top right\\
        						        & 6. That bookmark will change from outline state to filled state\\
        \hline
        Alternative flow  & No\\
        \hline
        Exception flow   & If the user hasn't logged in, the application will notify that "You have to login to perform this"\\
        \hline
  \end{tabular}
  \caption{Mark learned word}
  \label{tab:4-mark-learned-word}
\end{table}
\chapter{Upcoming plan}

TL;DR: Discuss shortly about the result, what we did not achieve and overview of the upcoming plan

\section{Translating system improvement}

\section{App design}

TODO: Put out the design from Figma
The current design of the application on the smartphone is considerably usable but not perfect, which means we need an upgrade in the design of the whole application in order to improve the user experience (UX) and user interface (UI).

\section{New feature}

To expand the group of people using this application, we will implement additional features into the main application to help ordinary people learn and know more about sign language. Those features are a sign language dictionary and a learning system that help people learn sign language more efficiently. We will discuss more these two primary functions in the below subsections.

\subsection{Sign language dictionary}


\subsection{Learning system}


\input{chapter/chapter6-proposedThesisChapters.tex}
\chapter{Summary}

This thesis applies image processing and artificial intelligence, whose purposes are to research and build a system that can translate sign language into Vietnamese using only a camera module and a smartphone.

So far, a sign language translating system has been a massive challenge to many scientists and engineers because of the complexity of sign language and the diversity of the way people use it around the world. Moreover, when we researched and built the system, there were a few similar systems, but they only translated the sign language alphabet.

In addition, talking about human values, this system can resolve the lack of sign-language translators in Vietnam. It, indeed, means that people having disabilities will have the chance to live, work and communicate like those who do not. They can have a better education as the teacher can understand their thoughts and connect more efficiently. They can have better health as the health force has the chance to know more about how they are, what they feel, which means we can provide them a better treatment for their problem. Their life will be easier as the surrounding people can get them and talk to them more clearly.

The deaf and mute are also a part of this world, a part of us, not apart from us. Therefore, we firmly believe the deaf and mute deserve to have the chance to speak up, be heard, be seen, and be acknowledged. With this application, we people can know each other and communicate fluently regardless of our level of knowledge of sign language. Ultimately, our bonds will grow more vital and more profound, which will lead to a better world for the entire human race.

Those human values emphasize the importance of this project in our world. Besides, the promising solution we presented throughout this proposal means it is possible to translate sign language with the current technologies. In the upcoming time, we have more resources to dedicate our time to completing our algorithm, which results in the higher accuracy of the translation process and completing the thesis thoroughly.


%-	Danh mục TL tham khảo
%-	Phụ lục (nếu có)
\bibliographystyle{unsrt}
\bibliography{ref}
\end{document}