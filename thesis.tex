% This is a copy version from https://github.com/thanhhungqb/thesis-template
% Please do not modified this project, when you want to start writing, make a clone of it for your own (Please read README.md)

\documentclass[12pt,a4paper,oneside]{book} % twoside for draf

%\usepackage{babel}
\usepackage[utf8]{vietnam}
\usepackage[utf8]{inputenc}
\usepackage[vietnamese,english]{babel}

\usepackage{tipa}
\usepackage{amssymb}
\usepackage{graphicx}
\usepackage{subcaption}
\usepackage{booktabs}
\usepackage{mathptmx}	% same Time New Roma
\usepackage{amsmath}
\usepackage{amssymb}
\usepackage{amsfonts}
\usepackage{sty/ipa}
%\renewcommand{\rmdefault}{phv} % Arial
%\renewcommand{\sfdefault}{phv} % Arial
\usepackage{array}
\newcolumntype{P}[1]{>{\centering\arraybackslash}p{#1}}
\newcolumntype{M}[1]{>{\centering\arraybackslash}m{#1}}
\usepackage{fancyhdr}
\usepackage{multirow}
\usepackage{algorithm2e}
\usepackage{hyperref}
\usepackage{float}

\usepackage{sty/bkthesis}

\usepackage{graphicx}
\graphicspath{ {image/} }
\setcounter{secnumdepth}{2}
\crname{GRADUATION THESIS PROPOSAL}
% \ctname{NHẬN DIỆN VẬT THỂ TRONG ẢNH\\NHẬN DIỆN HƯỚNG NHÌN TRONG ẢNH}
\ctname{USING MACHINE LEARNING METHODS\\IN TRANSLATING SIGN LANGUAGE\\INTO VIETNAMESE}
\cstuname{
Student: Võ Tuấn Khanh (1810220)\\Nguyễn Trí Nhân (1810390)}
\csSupervise{Assoc. Prof. Quan Thanh Tho}

\cttime{December 2021}

\thesislayout

\begin{document}
%-	Bìa cứng - màu xanh dương, chữ mạ vàng (xem mẫu đính kèm)
%-	Trang tên (tờ lót): chất liệu giấy, nội dung giống như bìa LV
%-	Ở gáy LV: in nhan đề LV (có thể in tóm tắt nếu nhan đề quá dài), size 15 – 17
%-	Phiếu Nhiệm vụ LV, chấm điểm Hướng dẫn & Phản biện (đã ký): nhận từ GVHD & GVPB sau khi bảo vệ (theo lịch hẹn).
%-	Lời cam đoan
%-	Lời cảm ơn/ Lời ngỏ
%-	Tóm tắt LV
%-	Mục lục
%-	Danh mục, bảng biểu, hình ảnh, ... (nếu có)
%-	Nội dung LV
%-	Danh mục TL tham khảo
%-	Phụ lục (nếu có)

\coverpage

\frontmatter


\begin{declaration}
	TODO: Viết sơ sơ về việc nội dung báo cáo không phải là false, ăn cắp này kia nọ

	ví dụ:

	Nhận diện hướng nhìn trong ảnh (Nhận diện vật thể trong ảnh) không phải là một đề tài mới nhưng vẫn là một thách thức bởi: trong các ứng dụng: việc nhận diện hướng nhìn của con người qua hình ảnh đòi hỏi kết quả chính xác cao, ở Việt Nam, hiện tại không  thực sự có nhiều nghiên cứu chuyên sâu về đề tài. Trong quá trình nghiên cứu đề tài có rất nhiều kiến thức không nằm trong chương trình giảng dạy ở bậc Đại học tuy vậy chúng tôi xin cam đoan đây là công trình nghiên cứu của riêng tôi dưới sự hướng dẫn của tiến sĩ Nguyễn Đức Dũng. Nội dung nghiên cứu và các kết quả đều là trung thực và chưa từng được công bố trước đây. Các số liệu được sử dụng cho quá trình phân tích, nhận xét được chính tôi thu thập từ nhiều nguồn khác nhau và sẽ được ghi rõ trong phần tài liệu tham khảo.
 	
 	Ngoài ra, tôi cũng có sử dụng một số nhận xét, đánh giá và số liệu của các tác giả khác, cơ quan tổ chức khác. Tất cả đều có trích dẫn và chú thích nguồn gốc.
 	
 	Nếu phát hiện có bất kì sự gian lận nào, tôi xin hoàn toàn chịu trách nhiệm về nội dung luận văn của mình. Trường đại học Bách Khoa thành phố Hồ Chí Minh không liên quan đến những vi phạm tác quyền, bản quyền do tôi gây ra trong quá trình thực hiện.

\end{declaration}

\begin{acknowledgments}
	TODO: Viết sau cùng -> về việc cảm ơn này kia

	ví dụ:

	Để hoàn thành kì đề cương luận văn này, tôi tỏ lòng biết ơn sâu sắc đến tiến sĩ Nguyễn Đức Dũng đã hướng dẫn tận tình trong suốt quá trình nghiên cứu.
	
	Chúng tôi chân thành cám ơn quý thầy, cô trong khoa Khoa Học Và Kỹ Thuật Máy Tính, trường đại học Bách Khoa thành phố Hồ Chí Minh đã tận tình truyền đạt kiến thức trong những năm chúng tôi học tập ở trường. Với vốn kiến thức tích lũy được trong suốt quá trình học tập không chỉ là nền tảng cho quá trình nghiên cứu mà còn là hành trang để bước vào đời một cách tự tin.

	Cuối cùng, tôi xin chúc quý thầy, cô dồi dào sức khỏe và thành công trong sự nghiệp cao quý.
	
\end{acknowledgments}
	
\begin{abstract}
	TODO: Viết sau cùng
	
	ví dụ:

	Nội dung chính của luận văn nhằm tìm hiểu, nghiên cứu xây dựng hệ thống nhận diện hướng nhìn thông qua ảnh chụp dựa trên những công trình, công nghệ mới được nghiên cứu và phát triển trong những năm gần đây của lĩnh vực Deep Learning. Trong quá trình nghiên cứu, tôi đã  tiến hành tổng hợp, đánh giá ưu và nhược điểm của cách phương pháp, công nghệ đã và đang được nghiên cứu, sử dụng. Tiếp cận vấn đề theo nhiều hướng khác nhau, tôi thực hiện một số phương pháp sử dụng học sâu (CNN) để phát hiện hướng nhìn của con người qua hình ảnh. Bên cạnh việc hoàn thành nội dung của đề tài, nhóm chúng tôi đã nghiên cứu thêm một số phần để từ đó đặt nền móng cho các nghiên cứu sau này. Phần còn lại của luận văn tập trung vào việc đánh giá mô hình, kết quả đạt được, đồng thời phân tích ưu nhược điểm của mô hình thực hiện và thảo luận những vấn đề mà mô hình còn gặp phải. Cuối cùng, nhóm chúng tôi đề xuất hướng phát triển tiếp theo của đề tài trong tương lai.
\end{abstract}	


\tableofcontents
% \listofsymbols
\listoffigures
\listoftables
%\listofalgorithms


\mainmatter

\fancyhead{}  % Clears all page headers and footers
%\rhead{\thepage}  % Sets the right side header to show the page number
%\lhead{}  % Clears the left side page header
%\fancyfoot[positions]{footer}
\renewcommand{\footrulewidth}{0.4pt}

\pagestyle{fancy}  % Finally, use the "fancy" page style to implement the FancyHdr headers
%

\chapter{Introduction}
	
\section{Problem statement}

"Each deaf person is a separate world, and they feel more self-deprecating and alone when they do not interact and share with others. They still have the desire to contribute to society", said Mr. Do Hoang Thai Anh, Vice Chairman of the Hanoi Deaf Association.

Language is a universal key that not only connects people but also builds up our society. Any disability that affects the ability to communicate is a significant disadvantage, especially for people with disabilities. They cannot integrate, have fun, learn, and communicate like ordinary people because they cannot express their thoughts, ideas, and desires to develop society as we do. That burden usually makes them fall into poverty, live a dependent life, and be exploited, apart from society. Hence, it is challenging for them to have beautiful lives.

In 2020, Vietnam had more than 2.5 million people who are deaf and mute, yet, only a tiny portion of them took part in education, had the chance to be understood, and integrated with society.

According to UNICEF, Households with members with disabilities are often poorer, children with disabilities are at risk of having less education than their peers, and employment opportunities for people with disabilities are also lower than those without disabilities. Even though people with disabilities are beneficiaries of the policy, and poverty is not a burden to accessing health facilities, very few people with disabilities (2.3\%) have access to functional rehabilitation services when being sick or injured. Besides, there still exist inequalities in living standards and social participation for people with disabilities [6]. Many organizations are founded to support, help, and create better living conditions for people with disabilities to develop. However, this work still has many difficulties and inadequacies as there is no formal school or class. Moreover, there is no specific profession for this group of people, and the number of translators who know sign language is insufficient, while they take an essential role in helping the people with disabilities connect with society.

A quote from Cavett Robert, "Life is a grindstone, and whether it grinds you down or polishes you up is for you and you alone to decide." However, it is challenging for these people to go to school and have an excellent education. They have their desires and dreams, but our resources and efforts are not enough to make them a polished grindstone. Furthermore, sign language shares the same property as any other spoken language; each different region and territory has a different way of expressing sign language. These unseen differences make communication, self-expression, and information exchange even more complex and challenging for humanity.

In short, we must admit that understanding and breaking the language barrier is extremely necessary and urgent because the deaf and mute, like many other ordinary people, deserve to be assisted, understood, and acknowledged. Furthermore, we believe our system is the resolve to problems of the deaf and hard of hearing.

\section{Goals of this thesis}

% Đề tài này hướng đến mục tiêu là nghiên cứu, hiểu và hiện thực một số giải pháp để có thể chuyển đổi từ ngôn ngữ ký hiệu sang tiếng Việt. Trong đó, hệ thống phải có khả năng thực hiện việc nhận video từ camera được gắn trên nón, thực hiện các giải thuật và xuất ra chữ trên màn hình điện thoại.
% 
% Từ mục tiêu tổng quát trên, tác giả sẽ lần lượt giải quyết các vấn đề sau để đưa ra một giải pháp thiết kế và hiện thực một kiến trúc hệ thống giải quyết được bài toán của đề tài:

This thesis aims to research, understand, and implement solutions to convert from sign language to Vietnamese. In particular, the system must receive a queue of images from the camera mounted on the hat, use the implemented algorithms to process and display text on the phone screen.

We can solve the above problem by breaking it down into smaller ones listed below. With each issue, we will give our solution and architect a system that can solve the whole problem of this thesis.
 
\begin{itemize}
	\item Search and collect data on sign language, conduct evaluation, classification, and normalization of data.
	\item Find out the approaches that have been implemented.
	\item Design architecture of the model
	\item Plan to implement, develop a sign language conversion system
	\item Build an application that users can utilize
\end{itemize}


\section{Scopes of this thesis}

% TLDR: In this case study, we will build a system including an app and camera module to translate at least 100 words from sign language into Vietnamese.

In this case study, we will build a system including an app and camera module to translate sign language into Vietnamese. Because of the limited time, the scope of the study is also limited as follows:

\begin{itemize}
	\item The system can only translate Vietnamese words
	\item The system can only recognize the words that it has been trained with
\end{itemize}

\section{Structure of this thesis proposal}

This proposal includes four sections and each will convey the related works and output when doing this thesis.

\begin{table}[H]
	\centering
	\begin{tabular}{ |c|p{13.5cm}| } 
		\hline
		Chapter          & Content                                                                                       \\
		\hline
		1                & A brief introduction about plan and objectives of thesis                                      \\
		\hline
		2                & Related works that had been done and how they help us in doing this thesis                    \\
		\hline
		\multirow{2}*{3} & Introduction of theoretical background as foundation knowledge that are applied in the thesis \\
		\hline
		4                & Solution and design approach for problem statement of this thesis                             \\
		\hline
		5                & Plan to finish this thesis in the upcoming time                                               \\
		\hline
		6                & Proposed chapters of the thesis                                                               \\
		\hline
		7                & Summary of this thesis proposal                                                               \\
		\hline
	\end{tabular}
	\caption{Structure of this thesis proposal}
\end{table}
\chapter{Theoretical Background}

Lorem ipsum

\chapter{Design and Solution}

\section{System Structure}

Lorem ipsum

\section{Detail Implementation}

Lorem ipsum

\chapter{Summary}

\section{Thesis Status}

Lorem ipsum

\section{Future Development}
 
Lorem ipsum


%bibliography{refs}{}
%bibliographystyle{plain}
%-	Danh mục TL tham khảo
%-	Phụ lục (nếu có)
\begin{thebibliography}{99}

\bibitem{9direction} 
Chi Zhang, Rui Yao, Jinpeng Cai
\textit{Efficient Eye Typing with 9 direction Gaze Estimation}.

\bibitem{appearance} 
Xucong Zhang, Yusuke Sugano, Mario Fritz, Andreas Bulling
\textit{Appearance-Based Gaze Estimation in the Wild}. 


\bibitem{eyeShapeRegistrationAndGazeEstimation} 
University of Cambridge, United Kingdom- Rendering of Eyes for Eye-Shape Registration and Gaze Estimation eww23 iccv2015
\\\url{https://www.cv-foundation.org/openaccess/content_iccv_2015/papers/Wood_Rendering_of_Eyes_ICCV_2015_paper.pdf}

\bibitem{Learninganappearancebasedgazeestimator} 
University of Cambridge and Carnegie Mellon University and Max Planck Institute for Informatics, Learning an appearance-based gaze estimator from one million synthesised images
\\\url{https://www.d2.mpi-inf.mpg.de/content/learning-appearance-based-gaze-estimator-one-million-synthesised-images}

\bibitem{AReviewandAnalysisofEyeGazeEstimation} 
Anuradha Kar and Peter M. Corcoran, A Review and Analysis of Eye-Gaze Estimation Systems Algorithms and Performance Evaluation Methods in Consumer Platforms
\\\url{https://www.semanticscholar.org/paper/A-Review-and-Analysis-of-Eye-Gaze-Estimation-and-in-Kar-Corcoran/ae0a0ee1c6e2adcddffebf9b0e429a25b7d9c0e1}


\bibitem{tangconv}
\url{https://developer.apple.com/library/content/documentation/Performance/Conceptual/vImage/ConvolutionOperations/ConvolutionOperations.html}

\bibitem{lenet5}
Y. Lecun, L.Boutou, and Y.Bengio, Gradient-based learning applied to document recognition, Proceedings of the IEEE, vol. 88, no. 11, pp. 2278 – 2324, Nov. 1998.

\bibitem{maxpool}
Denny Britz,
\url{http://www.wildml.com/2015/11/understanding-convolutional-neural-networks-for-nlp/}

\bibitem{cnn}
Brandon Rohrer, \url{http://brohrer.github.io/how_convolutional_neural_networks_work.html}

\bibitem{fullconnect}
Trần Thế Anh, 
\url{http://labs.septeni-technology.jp/technote/ml-20-convolution-neural-network-part-3/}
\bibitem{ptha}
Lương Quốc An, 
\url{http://nhiethuyettre.net/mang-no-ron-tich-chap-convolutional-neural-network/}

\bibitem{inception}
\url{https://leonardoaraujosantos.gitbooks.io/artificial-inteligence/content/googlenet.html}

\bibitem{softmax}
Giáo trình Mạng neural, Tác giả: Phan Văn Hiền – Trường Đại học Bách khoa Đà Nẵng, 2013

\bibitem{alexnet} Aarshay Jain, 
\url{https://www.analyticsvidhya.com/blog/2016/04/deep-learning-computer-vision-introduction-convolution-neural-networks/}
\bibitem{dataset}
\url{https://www.mpi-inf.mpg.de/departments/computer-vision-and-multimodal-computing/research/gaze-based-human-computer-interaction/its-written-all-over-your-face-full-face-appearance-based-gaze-estimation/}

\bibitem{}
\url{https://www.tensorflow.org/versions/r0.12/get_started/basic_usage}

\bibitem{gglenet}
Christian Szegedy, Wei Liu, Yangqing Jia, Pierre Sermanet, Scott Reed, Dragomir Anguelov, Dumitru Erhan, Vincent Vanhouke, Andrew Rabinovich. \textit{Going deeper with convolutions}

\bibitem{renset}
Kaiming He, Xiangyu Zhang, Shaoqing Ren, Jian Sun \textit{Deep Residual Learning for Image Recognition}

\bibitem{tensor} Trần Thế Anh, 
\url{http://labs.septeni-technology.jp/technote/ml-18-convolution-neural-network-part-1/}

\bibitem{mangcnn}
\url{https://www.kernix.com/blog/a-toy-convolutional-neural-network-for-image-classification-with-keras_p14}
\bibitem{GazeCaptureEyeTracking}
Kyle Krafka- Aditya Khosla- Petr Kellnhofer- Harini Kannan- Suchendra Bhandarkar- Wojciech Matusik- Antonio Torralba, Eye Tracking for Everyone
\url{http://gazecapture.csail.mit.edu/}

\bibitem{GazeCapturegit}
Kyle Krafka and Aditya Khosla and Petr Kellnhofer and Harini Kannan and Suchendra Bhandarkar and Wojciech Matusik and Antonio Torralba, Eye Tracking for Everyone Code Dataset and Models
\url{https://github.com/CSAILVision/GazeCapture}

\bibitem{eyetrackingapplication}
\url{https://medium.com/@taolu_99738/developing-of-eye-tracking-application-for-smartphone-b875c50ee0c3}

\end{thebibliography}
\end{document}